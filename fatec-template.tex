%%%% fatec-article.tex, 2024/03/10

%% Classe de documento
\documentclass[
landscape,
  a4paper,%% Tamanho de papel: a4paper, letterpaper (^), etc.
  12pt,%% Tamanho de fonte: 10pt (^), 11pt, 12pt, etc.
  english,%% Idioma secundário (penúltimo) (>)
  brazilian,%% Idioma primário (último) (>)
]{article}

%% Pacotes utilizados
\usepackage[]{fatec-article}
\usepackage{setspace}

%% Processamento de entradas (itens) do índice remissivo (makeindex)
%\makeindex%

%% Arquivo(s) de referências
%\addbibresource{fatec-article.bib}

%% Início do documento
\begin{document}

% Seções e subseções
%\section{Título de Seção Primária}%

%\subsection{Título de Seção Secundária}%

%\subsubsection{Título de Seção Terciária}%

%\paragraph{Título de seção quaternária}%

%\subparagraph{Título de seção quinária}%

%\section*{Diário de Bordo}%
\section*{Instruções para o preenchimento}
\doublespacing
\begin{enumerate}
    \item O Diário de Bordo é usado para registrar atividades, progressos, ideias e desafios enfrentados em um projeto ou durante a rotina de trabalho. Serve como um registro cronológico e detalhado das operações diárias, facilitando a organização e o acompanhamento das tarefas.
    \doublespacing
    \item Durante o registro das atividades deve-se incluir detalhes como datas, horários, descrições de tarefas, nomes de participantes e observações relevantes.  Esta documentação contínua ajuda na avaliação do progresso de projetos ou atividades, permitindo ajustes e melhorias contínuas nos processos.
    \doublespacing
    \item Para evidenciar a realização das tarefas, você poderá utilizar a criação de anexos para adicionar anotações, fotos, prints, questionários, entre outros.
\end{enumerate}

\break

 \begin{table}[]
\centering
\begin{tabular}{|l|l|l|l|l|}
\hline
Nome da Atividade           & Data de início & Data de término & Responsável pela atividade & Descrição da atividade realizada                    \\ \hline
Entrega de Temas do PI      & 31/08/2024     & 03/09/2024      & Arthur                     & Atualização e entrega do tema da equipe             \\ \hline
Canvas e Lean Canvas        & 06/09/2024     & 06/09/2024      & Giovana                    & Realização do Canvas e Lean Canvas para a segunda   \\ 
                            &                &                 &                            & etapa do inova.                                     \\ \hline
Reestruturação do grupo     & 09/09/2024     & 09/09/2024      & Todos os membros           & Anunciamento da saída de Arthur e Giovana do grupo  \\ \hline
Diagrama de classe do       & 15/09/2024     & 15/09/2024      & Lucas, Éric                & Começo do diagrama de classe do projeto             \\ 
projeto                     &                &                 &                            &                                                     \\ \hline
Problema, público e solução & 20/09/2024     & 20/09/2024      & Kaique, Lucas e Éric       & Definição do problema, público e solução do projeto \\ 
                            &                &                 &                            & para o vitrine.                                     \\ \hline
Validação do problema       & 24/09/2024     & 27/09/2024      & Kaique                     & Feita uma pesquisa de campo com agentes de endemias \\ 
                            &                &                 &                            & para a validação da idéia do projeto.               \\ \hline
Edição do artigo científico & 05/10/2024     & 08/10/2024      & Kaique, Lucas              & Alteração do resumo, introdução e objetivo do artigo\\ \hline
Alteração da identidade     & 22/10/2024     & 22/10/2024      & Kaique                     & Edição da introdução do manual de identidade visual \\ 
visual do Epidemix          &                &                 &                            & Epidemix                                            \\ \hline
Definição de papéis         & 22/10/2024     & 22/10/2024      &Todos os membros            & Foram feitas as definições de qual membro do projeto\\ 
                            &                &                 &                            & iria ficar com cada função                          \\ \hline
Edição do artigo científico & 29/10/2024     & 18/11/2024      & Lucas, Kaique              & Edição da metodologia, estado da arte e dos         \\ 
                            &                &                 &                            & resultados preliminares                             \\ \hline
Aplicação JAVA              & 08/11/2024     & 18/11/2024      & Éric                       & Começo e final da fabricação da aplicação Java      \\ \hline
Aplicação Node.JS           & 11/11/2024     & 18/11/2024      & Kaique                     & Começo e final da fabricação da aplicação web em    \\ 
                            &                &                 &                            & Node.JS                                             \\ \hline
                            &                &                 &                            &                                                     \\ \hline
                            &                &                 &                            &                                                     \\ \hline
                            &                &                 &                            &                                                     \\ \hline
                            &                &                 &                            &                                                     \\ \hline
\end{tabular} 
\end{table}



\end{document}